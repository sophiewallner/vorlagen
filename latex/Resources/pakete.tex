%  ------------------------------ Allgemein ------------------------------
% Querformat:
% \usepackage[landscape]{geometry}

% deutsche Sprache bei Silbentrennung und Datum:
\usepackage[ngerman]{babel}

% keine Einrückung bei einem neuen Absatz
\setlength{\parindent}{0pt}

% Festlegung der Breite und Gestaltung eines Seitenrandes; Hinzufügen mit \marpar{}
\setlength{\marginparwidth}{2,3cm}
\newcommand{\marpar}[1]{\marginpar{\scriptsize \color{black!60} \flushleft {#1}}}

% pimpt enumerate auf (optionales Argument liefert Nummerierung):
\usepackage{enumerate}
\usepackage{enumitem}
% Zeilenabstand:
%\usepackage[
%  % nur eine Möglichkeit auswählen:
%  singlespacing
%  %onehalfspacing
%  %doublespacing
%]{setspace}

% Hinzufügen mit \todo{} oder auch \todo[inline]{}
\usepackage[
ngerman,
textwidth=2cm,
textsize=small,
backgroundcolor=red!20,
linecolor=black,
]{todonotes}

% weiteres todo, dass sich nur auf Formatierung bezieht:
\newcommand{\ftodo}[1]{\todo[color=blue!20,linecolor=black]{#1}}

% Blindtext
\usepackage{blindtext}
\usepackage{rotating}
\usepackage[list=true, font=large, labelfont=bf, 
labelformat=brace, position=top]{subcaption}
% Wörter rot markieren
\newcommand{\red}[1]{\textcolor{red}{#1}}

% Abstract
\usepackage{abstract}
\addto\captionsngerman{
\renewcommand{\abstractname}{Abstract}
}

% Danksagung
\newcommand\danke{Danksagung}
\newenvironment{danksagung}
  {
  \renewcommand\abstractname{\danke}
  \begin{abstract}
  }
  {\end{abstract}
  }

% für schlaue Zitate:
\usepackage{epigraph}
\setlength{\epigraphwidth}{0.42\textwidth}

% noch kleiner als tiny
\usepackage{lmodern}
\newcommand{\supertiny}{\fontsize{4}{4}\selectfont}

% Ermöglicht durch \begin{linenumbers} Zeilennummern anzuzeigen:
%\usepackage{lineno}

\usepackage{pifont}% http://ctan.org/pkg/pifont
\newcommand{\cmark}{\ding{51}}%
\newcommand{\xmark}{\ding{55}}%

%  ------------------------------ Mathematik ------------------------------
\usepackage[
  leqno,%% Nummerierung von Gleichungen links:
  fleqn, %% Ausgabe von Gleichungen linksbündig:
]{mathtools}

\usepackage{amsmath,amssymb,dsfont,upgreek}

% Symbole für theoretische Informatik (z.B. \lightning):
\usepackage{stmaryrd}

% für Matrizen
\usepackage{array}

% Betragsstriche und vertikale Striche mit etwas mehr Raum
\newcommand\abs[1]{\vert #1 \vert}
\newcommand{\verts}{\, \vert \,}

% Multimenge
\newcommand{\multiset}[1]{ \{ \vert #1 \vert \} }

% O-Notation
\newcommand{\BigO}{\mathcal{O}}

% varphi als Standard-Phi und normales Phi (zur Bezeichnung für Isomorphismen)
\newcommand{\iso}{\upphi}
\renewcommand{\phi}{\varphi}

% ------------------------------ Definitionen und Sätze ------------------------------ 
\usepackage{amsthm}

\newtheoremstyle{defpropumgebung}
{9pt}% measure of space to leave above the theorem. E.g.: 3pt
{9pt}% measure of space to leave below the theorem. E.g.: 3pt
{}% name of font to use in the body of the theorem
{}% measure of space to indent
{\bfseries}% name of head font
{\smallskip}% punctuation between head and body
{\newline}% space after theorem head; " " = normal interword space
{}% Manually specify head

\theoremstyle{defpropumgebung}
\newtheorem{definition}{Definition}
\newtheorem{proposition}{Proposition}
% \newtheorem{beispiel}{Beispiel}
\newtheorem{lemma}{Lemma}

\newtheoremstyle{beweisumgebung}
{9pt}% measure of space to leave above the theorem. E.g.: 3pt
{9pt}% measure of space to leave below the theorem. E.g.: 3pt
{}% name of font to use in the body of the theorem
{}% measure of space to indent
{\itshape}% name of head font
{}% punctuation between head and body
{0.5em}% space after theorem head; " " = normal interword space
{}% Manually specify head

\theoremstyle{beweisumgebung}
\newtheorem*{beweis}{Beweis:} % * = ohne Nummerierung für Beweise
\newcommand{\bewiesen}{
    \hfill
    $ \square $  
}

\newcommand{\gdw}{\text{ gdw. }}

\newtheoremstyle{beispielumgebung}
{9pt}% measure of space to leave above the theorem. E.g.: 3pt
{15pt}% measure of space to leave below the theorem. E.g.: 3pt
{}% name of font to use in the body of the theorem
{}% measure of space to indent
{\bfseries}% name of head font
{\smallskip}% punctuation between head and body
{\newline}% space after theorem head; " " = normal interword space
{}% Manually specify head

\theoremstyle{beispielumgebung}
\newtheorem{beispiel}{Beispiel}

% Subscript Größe anpassen
\usepackage{scalerel}
\newcommand{\N}{\scaleto{N}{3.2pt}}
\newcommand{\C}{\scaleto{C}{3.2pt}}
\newcommand{\U}{\scaleto{U}{3.2pt}}
\renewcommand{\S}{\scaleto{S}{3.2pt}}
\newcommand{\Sp}{\scaleto{S'}{3.2pt}}
\newcommand{\1}{\scaleto{1}{3.2pt}}
\newcommand{\2}{\scaleto{2}{3.2pt}}
\newcommand{\3}{\scaleto{2}{3.2pt}}

%  ------------------------------ Textgestaltung ------------------------------
\usepackage{microtype}
\usepackage{ifxetex}
\ifxetex{}
\usepackage[MnSymbol]{mathspec}
  
\defaultfontfeatures{
    Ligatures=TeX,
    Scale=MatchLowercase,
}
% Hauptschriftart, Matheschriftart, Theoremschriftart
\setmainfont[]{Minion Pro}
\setmathfont(Digits,Latin,Greek)[]{Minion Pro}
\setmathrm{Minion Pro}

\setallsansfonts[
    BoldFont = {Gill Sans},
  ]{Gill Sans}
\newfontface\titlepagefont{Gill Sans}
\else
  \usepackage[utf8]{inputenc}
\fi

% Unterstreichungen (\uline, \uuline, \sout: durchgestrichen, \uwave):
%\usepackage{ulem}

% Aufzählungszeichen kleiner als herkömmliche Bullets
\renewcommand\labelitemi{\boldmath $ \cdot $}

% ------------------------------ Tabellen ------------------------------	
% kann komplexe Linien in Tabellen ziehen:
\usepackage{hhline}

% mehrseitige Tabellen:
%\usepackage{longtable}

% Tabellen mit gedehnten Spalten:
%\usepackage{tabularx}

% Schönere Tabellen:
% - \toprule[(Dicke)], \midrule[(Dicke)], \bottomrule[(Dicke)]
% - \addlinespace: Extrahöhe zwischen Zeilen
\usepackage{booktabs}

% Kann descriptions auf die selbe Höhe bringen:
%\usepackage{enumitem}

% bietet gestrichelte vert. Linien in Tabellen (':')
%\usepackage{arydshln}

% um in Tabellen eine Zelle über mehrere Zeilen laufen zu lassen:
%\usepackage{multirow}


% ------------------------------ Links ------------------------------
% Liefert Hyperlinks (\hyperref, \url, \href}
\usepackage{hyperref}
\hypersetup{
  colorlinks=false,
  linkcolor=black,
  urlcolor=blue,
}

% ------------------------------ Algorithmen/Code ------------------------------
%\usepackage{listings}

% Algorithmen und Pseudocode:
\usepackage{algorithm}
\usepackage{algorithmic}
\floatname{algorithm}{Algorithmus}
\renewcommand{\algorithmicrequire}{\textbf{Eingabe:}}
\renewcommand{\algorithmicensure}{\textbf{Ausgabe:}}

% Verbessert fließende Objekte
%
%     \begin{figure}[H]
%     ...
%     \end{figure}
% wird’s auch wirklich _hier_ gesetzt
%\usepackage{float}

% SI-Einheiten mittels \si{}:
%\usepackage[mode=text]{siunitx}
%\sisetup{%
%  output-decimal-marker={,},
%  per-mode=fraction,
%  exponent-product=\cdot,
%}
%\DeclareSIUnit\cal{cal}
%\DeclareSIUnit\diopter{dpt}
%\DeclareSIUnit\fahrenheit{F}
%\DeclareSIUnit\molar{\textsc{m}}
%\DeclareSIUnit\pH{pH}
%\DeclareSIUnit\gewprozent{Gew\%}
%\DeclareSIUnit\poise{P}

%\usepackage{titlesec}
%\titleformat*{\paragraph}{\itshape\mdseries} % chktex 6
% \titleformat{\section}
%   {\sffamily}{\thesection}{1em}{}

% ein Eintrag in einer description-Liste wird in ganz normaler Schrift angezeigt (kein
% sans-serif, kein fett):
\renewcommand{\descriptionlabel}[1]{\hspace{\labelsep}#1}


% ------------------------------ Kopf- und Fußzeile ------------------------------
\usepackage{fancyhdr}
%% Seitenstil ist natürlich fancy:
%\pagestyle{fancy}
%% alle Felder löschen:
%\fancyhf{}
%% Veranstaltung:
%%\fancyhead[L]{}
%% Seriennummer:
%%\fancyhead[C]{}
%% Name und Matrikelnummer:
%%\fancyhead[R]{}
%%\fancyfoot[L]{}
%\fancyfoot[C]{\thepage}
%%\fancyfoot[C]{\thepage\,/\,\pageref{LastPage}}
%% Linie oben/unten:
%\renewcommand{\headrulewidth}{0.0pt}
%\renewcommand{\footrulewidth}{0.0pt}

% besondere Zeichen:
%\usepackage{pifont}
%\newcommand{\cmark}{\ding{51}}%
%\newcommand{\xmark}{\ding{55}}%
%\newcommand{\richtig}{\textcolor{ForestGreen}{\cmark}}
%\newcommand{\falsch}{\textcolor{BrickRed}{\xmark}}


% ------------------------------ Abbildungen ------------------------------
%TIKZ und pgf für grafische Elemente
\usepackage{pgfplots}
\usepackage{tikz}
\usetikzlibrary{arrows,shapes,snakes,automata,backgrounds,petri,arrows.meta,positioning}
% Verbessert den Satz von Abbildungsüberschriften:
\usepackage{caption}
\tikzset{
  transition/.style={shape=rectangle,draw=black,minimum size=0.7cm,inner sep=0.3pt},
  >=latex,
  greentoken/.style={shape=circle,draw=black,fill=black!30!green,minimum size=1.5mm,inner sep=0.3pt},
  bluetoken/.style={shape=circle,draw=black,fill=black!30!blue,minimum size=1.5mm,inner sep=0.3pt}, 
  redtoken/.style={shape=circle,draw=black,fill=black!30!red,minimum size=1.5mm,inner sep=0.3pt},
  blacktoken/.style={shape=circle,draw=black,fill=black,minimum size=1.5mm,inner sep=0.3pt},
  state/.style={circle,fill,inner sep=1pt,label=#1\strut},
  % stateb/.style={circle,fill,inner sep=1pt,label=above:#1\strut},
  arrow/.style={-latex, shorten >=1ex, shorten <=1ex}
}
% für Zeichnungen:
%\usetikzlibrary{%
%  backgrounds,
%  %mindmap,
%  %shapes.geometric,
%  %shapes.symbols,
%  %shapes.misc,
%  %shapes.multipart,
%  %positioning,
%  %fit,
%  calc,
%  arrows,
%  %automata,
%  %trees,
%  %decorations.pathreplacing,
%  %circuits.ee.IEC,
%  intersections,
%  through,
%}

%\pgfplotsset{compat=1.16}

% eigens gebaute Lochmarken:
%\usepackage{eso-pic}
%\AddToShipoutPicture*{
%\put(\LenToUnit{0mm},\LenToUnit{228.5mm})
%{\rule{\LenToUnit{20pt}}{\LenToUnit{0.5pt}}}
%\put(\LenToUnit{0mm},\LenToUnit{68.5mm})
%{\rule{\LenToUnit{20pt}}{\LenToUnit{0.5pt}}}
%}