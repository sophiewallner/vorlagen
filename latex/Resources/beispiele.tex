\section{Einleitung}
\label{sec:einleitung}

\blindtext
Das gleiche Prinzip wird auch in \cite{clarke} beschrieben.


% eigene Notizen, die im Laufe der Arbeit ein- oder abgearbeitet werden sollen
\todo{Ebenfalls können TODOs hinzugefügt werden}

% Definitionen (mit Absatz) und Sätze (ohne Absatz)
\begin{definition}
	Das \textit{Definiendum} wird kursiv geschrieben. Ansonsten kann hier alles stehen was man mag.
\end{definition}

\section{Mathematik}
\label{sec:mathematik}

Ein bisschen Mathematik im Text ist z.\,B. $x^2 + y^2 = z^2$.
Es geht aber auch anders:
% align* verhindert die Nummerierung der Zeilen (sofern keine Referenzierung dieser nötig)
\begin{align*}
  {(x+y)}^n & = \sum\limits_{k=0}^{n} \binom{n}{k} x^k y^{n-k}
  & \int\limits_{-\infty}^{+\infty} e^{-x^2} dx & = \sqrt{\pi}
\end{align*}

\section{Abbildungen und Tabellen}
\label{sec:abbildungen}

\subsection{Matrix}
In der Mathematik-Umgebung können Matrizen eingebunden werden. Hierbei gibt es verschiedene Darstellungsmöglichkeiten, wie: 
$
\left(
	\begin{array}{c c c}
	1 & 2 & 3 \\
	2 & 1 & 3 \\
	3 & 2 & 1 \\
	\end{array}
\right)
 \text{ oder }
\left[
\begin{array}{c c c}
1 & \cdots & 3 \\
2 & \ddots & 3 \\
3 & \cdots & 1 \\
\end{array}
\right]
$

\subsection{Tabellen}
\begin{minipage}{0.5\textwidth}
\begin{center}
	\begin{tabular}{|c c c|}
		\hline
		Bezeichnung & Merkmal & Rating \\ \hline
		Igel & Stacheln & 3/5 \\
		Bär & Tatzen & 4/5 \\
		Hund & Hecheln & 2/5 \\
		\hline
	\end{tabular}
\end{center}
\end{minipage}
\begin{minipage}{0.5\textwidth}
\begin{center}
	\begin{tabular}{||cc||c|c||}
		\hhline{|t:==:t:==:t|}
		a&b&c&d\\
		\hhline{|:==:|~|~||}
		1&2&3&4\\
		\hhline{#==#~|=#}
		i&j&k&l\\
		\hhline{||--||--||}
		w&x&y&z\\
		\hhline{|b:==:b:==:b|}
	\end{tabular}
\end{center}
\end{minipage}

\subsection{Petrinetze}
\begin{figure}
	\begin{center}
		\begin{tikzpicture}[
		whitetoken/.style={shape=circle,draw=black,fill=white,minimum size=1.1mm,inner sep=0.3pt}, 
		greytoken/.style={shape=circle,draw=black,fill=black!30,minimum size=1.1mm,inner sep=0.3pt},
		guard/.style={shape=circle,minimum size=0mm,inner sep=0pt},
		netname/.style={shape=circle,minimum size=0mm,inner sep=0pt}]
		%net label
		\node[netname] at (2,1){$ N_C $};
		%places of coloured net with coloured token
		\node[place,label=above:$p_{C_1}$]	(p1) {};
		\node[whitetoken] at(-0.12,0) {\tiny1};
		\node[whitetoken] at(0.12,0) {\tiny1};
		\node[place,label=above:$p_{C_2}$]	at (0,-2) (p2) {};
		\node[whitetoken] at(-0.12,-2) {\tiny1};
		\node[whitetoken] at(0.12,-2) {\tiny1};
		\node[place,label=above:$p_{C_3}$]	at (0,-4) (p3) {};
		\node[whitetoken] at(-0.12,-4) {\tiny1};
		\node[whitetoken] at(0.12,-4) {\tiny1};
		\node[place,label=above:$p_{C_4}$] at(4,-1)	(p4){};
		\node[place,label=above:$p_{C_5}$] at(4,-3)	(p5){};
		%transition tc1
		\node[transition, label=above:$t_{C_1}$] at(2,-1) (t1){}
		edge[pre] node[above,near end]{$x$} (p1)
		edge[pre] node[above,very near end]{$y$}(p2)
		edge[pre] node[above,very near end]{$z$}(p3)
		edge[post] node[above]{$x$}(p4);
		\node[guard] at(3,-1.5){$g_{t_{c_1}}$};
		%transition tc2
		\node[transition, label=above:$t_{C_2}$] at(2,-3) (t2){}
		edge[pre] node[below,very near end,xshift=-2pt]{$2x$}(p1)
		edge[pre] node[below,very near end]{$2y$}(p2)
		edge[pre] node[below,near end]{$2z$}(p3)
		edge[post] node[above]{$2x$}(p5);
		\node[guard] at(3,-3.5){$g_{t_{c_2}}$};
		\end{tikzpicture}
	\end{center}
	\caption{gefärbtes Petri Netz mit blockierenden Guards}
	\label{fig:petrinet}
\end{figure}

\subsection{Pseudocode}
\begin{algorithm}
	\caption{Berechne $y = x^n$}
	\begin{algorithmic}
		\REQUIRE $n \geq 0 \vee x \neq 0$
		\ENSURE $y = x^n$
		\STATE $y \leftarrow 1$
		\IF{$n < 0$}
		\STATE $X \leftarrow 1 / x$
		\STATE $N \leftarrow -n$
		\ELSE
		\STATE $X \leftarrow x$
		\STATE $N \leftarrow n$
		\ENDIF
		\WHILE{$N \neq 0$}
		\IF{$N$ ist gerade}
		\STATE $X \leftarrow X \times X$
		\STATE $N \leftarrow N / 2$
		\ENDIF
		\ENDWHILE
	\end{algorithmic}
\end{algorithm}